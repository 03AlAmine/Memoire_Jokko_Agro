\chapter*{Liste des acronymes}  % Section sans numéro
\addcontentsline{toc}{chapter}{Liste des acronymes}  % Ajouter à la table des matières

\begin{flushleft}
    \begin{itemize}
        \item \textbf{SSH} : Secure Shell  
        \item \textbf{YUM} : Yellowdog Updater
        \item \textbf{CD} : Déploiement Continu  
        \item \textbf{AWS} : Amazon Web Services  
        \item \textbf{CI} : Intégration Continue  
        \item \textbf{TSDB} : Time Series Database
        \item \textbf{CLI} : Command Line Interface 
        \item \textbf{RPM} : Red Hat Package Manager 
        \item \textbf{CPU} : Central Processing Unit  
        \item \textbf{URL} : Uniform Resource Locator
        \item \textbf{HTTP} : HyperText Transfer Protocol, 
        \item \textbf{TCP} : Transmission Control Protocol
        \item \textbf{Amazon EC2} : Amazon Elastic Compute Cloud  
        \item \textbf{HTTPS} : HyperText Transfer Protocol Secure
        \item \textbf{API} : Interfaces de programmation d'applications  

    \end{itemize}
    \end{flushleft}
    
    % Configuration de l'image de fond

    \chapter*{Introduction}  % Section sans numéro
    \EnableIntroBackground % Activer le fond
    \addcontentsline{toc}{chapter}{Introduction}  % Ajouter à la table des matières
    « Plutôt qu'un marché, DevOps est avant tout une philosophie, un changement culturel qui associe les mondes du développement et des opérations », explique Laurie Wurster, directrice de recherche chez Gartner. Depuis sa naissance en 2009, le mouvement DevOps a révolutionné la façon dont les entreprises conçoivent, développent et déploient leurs applications.\\

    Des entreprises emblématiques de la Silicon Valley, telles que Google, Microsoft, Amazon, Netflix et LinkedIn, ont rapidement adopté DevOps pour améliorer leurs performances, stimuler leur capacité d’innovation et renforcer leur compétitivité. \\
    
    Dans cet écosystème, des outils clés jouent un rôle central pour automatiser, surveiller et gérer les performances des applications dans des environnements complexes et dynamiques. Ces technologies facilitent non seulement le déploiement des applications, mais elles garantissent aussi leur disponibilité, leur scalabilité et la qualité du service.\\
    
    L’intégration de ces outils dans une infrastructure moderne assure une gestion optimale des applications tout en offrant des capacités avancées de surveillance. Cela permet une réactivité accrue face aux problèmes de performance, tout en soutenant la stabilité des systèmes et en favorisant l'innovation et la croissance des entreprises. \\
    
    Prometheus, par exemple, interroge régulièrement les points de terminaison d’une application pour collecter des métriques essentielles à son bon fonctionnement. Cependant, la collecte des métriques n’est qu’une première étape. Leur interprétation et leur visualisation sont tout aussi importantes. 
    
    Grafana joue ici un rôle crucial en tant que tableau de bord visuel, permettant de surveiller, analyser et présenter ces métriques de manière intuitive. Sa flexibilité et ses nombreuses options de personnalisation en font un outil incontournable pour les équipes DevOps.
    
    Enfin, la gestion des alertes via Alertmanager simplifie la supervision en connectant les alertes aux canaux de communication tels que les emails ou Slack. Cette automatisation renforce la réactivité des administrateurs, qui peuvent agir rapidement pour résoudre les problèmes et minimiser les interruptions de service.
    
